\usepackage{color}
\documentclass{beamer}

\usepackage{amsthm}
\usepackage{amssymb}
\usepackage{graphicx}

\newtheorem{prop}{Proposition}
\newtheorem{cor}[prop]{Corollary}
\newtheorem{thm}[prop]{Theorem}
\newtheorem{lem}[prop]{Lemma}
\newtheorem{conj}[prop]{Conjecture}
\newtheorem*{def concordance}{Definition}

\theoremstyle{definition}
\newtheorem*{defn}{Definition}
\newtheorem*{exam}{Example}
\newtheorem*{remark}{Remark}
\newtheorem*{caut}{Caution!}
\newtheorem*{ques}{Question}

\newtheorem*{thma}{Main Theorem}
\newtheorem*{thmb}{Theorem B}

\newcommand{\inv}{^{-1}}
\newcommand{\p}{\pi_1}
\newcommand{\del}{\partial}
\renewcommand{\a}{\alpha}
\renewcommand{\b}{\beta}
\newcommand{\lk}{{\mathrm{lk}}}
\newcommand{\Gn}[1]{G^{(#1)}}
\newcommand{\pr}[2]{\p({#1})_r^{(#2)}}
\newcommand{\Gr}[2]{{#1}_r^{(#2)}}
\renewcommand{\H}[2]{H_{#1}\left({#2}\right)}
\newcommand{\pn}[2]{\p\left({#1}\right)^{({#2})}}
\newcommand{\F}{{\mathcal{F}}}
\newcommand{\C}{{\mathcal{C}}}
\newcommand{\R}{{\mathbb{R}}}
\newcommand{\Z}{{\mathbb{Z}}}
\newcommand{\Q}{{\mathbb{Q}}}
\newcommand{\G}{{\mathcal{G}}}
\newcommand{\dg}{{\mathrm{dg}}}
\newcommand{\rank}{{\mathrm{rank\ }}}
\newcommand{\T}{{\mathcal{T}}}
\newcommand{\im}{{\mathrm{im\ }}}
\newcommand{\spin}{{\mathrm{spin}}}
\newcommand{\spinc}{{\mathrm{spin^c}}}
\newcommand{\N}{{\mathcal{N}}}
\newcommand{\s}{{\mathfrak{s}}}
\newcommand{\comfig}{{/Users/peter/Documents/CommonFigures/pdf}}
\renewcommand{\S}{{\mathcal{S}}}
\newcommand{\cfkinf}{CFK^\infty}


\renewcommand{\P}{{\mathcal{P}}}

% Beamer options
% ====================================================================
% ====================================================================
\usetheme{Dresden}
%\usecolortheme{wolverine}
\setbeamertemplate{theorems}[numbered]
\usecolortheme{orchid}
\usefonttheme{serif}
\begin{document}

\title[Linearly Independent TOP Slice knots]{A class of linearly independent topologically slice knots}
\author{team}
\date{July 29}
\institute{Columbia}

\frame{\titlepage}

\frame{
  \framtetitle{The Concordance Group}
  Recall that two knots $K$ and $J$ are {\bf isotopic} if they can be smoothly deformed into each other. Or formally, $\exists$ a homotopy $\kappa:S^{1}\times[0,1]\longrightarrow \mathbb{R}^3$ such that $\kappa \times{0}=K$, and $\kappa \times{1}=J$, and $\kappa \times {t}$ is a smooth embedding, $\forall t\in [0,1]$. Now, we generalise this idea by considering the deformation occured in an extra dimension: we say knots $K$ and $J$ are {\bf topological concordant/(smoothly) concordant} if they can be topologically/smoothly (resp.) deformed into each other in four dimensions. The rigorous definition is: 
  \begin{def concordance}
    Oriented knots K and J are {\bf topologically concordant/(smoothly) concordant} if $\exists$ annulus $A = S^{1}\times[0,1]$ topologically/smoothly (resp.) embedded into $S^{3}\times [0,1]$, such that $A\cap (S^{3}\times {0})= K\subseteq (S^{3}\times {0})$, and  $A\cap (S^{3}\times {1})= -J\subseteq (S^{3}\times {1})$.
    \\
    Denote $K \sim_{top}J$ for topological concordance, and $K \sim_{c}J$ for (smooth) concordance).
  \end{def concordance}
}
\end{document}
	
