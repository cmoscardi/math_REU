\documentclass{beamer}
\usepackage{amsthm}
\usepackage{amssymb}
\usepackage{graphicx}
\usepackage{color}

\newtheorem{prop}{Proposition}
\newtheorem{cor}[prop]{Corollary}
\newtheorem{thm}[prop]{Theorem}
\newtheorem{lem}[prop]{Lemma}
\newtheorem{conj}[prop]{Conjecture}
\newtheorem*{def concordance}{Definition}

\theoremstyle{definition}
\newtheorem*{defn}{Definition}
\newtheorem*{exam}{Example}
\newtheorem*{remark}{Remark}
\newtheorem*{caut}{Caution!}
\newtheorem*{ques}{Question}

\newtheorem*{thma}{Main Theorem}
\newtheorem*{thmb}{Theorem B}

\newcommand{\inv}{^{-1}}
\newcommand{\p}{\pi_1}
\newcommand{\del}{\partial}
\renewcommand{\a}{\alpha}
\renewcommand{\b}{\beta}
\newcommand{\lk}{{\mathrm{lk}}}
\newcommand{\Gn}[1]{G^{(#1)}}
\newcommand{\pr}[2]{\p({#1})_r^{(#2)}}
\newcommand{\Gr}[2]{{#1}_r^{(#2)}}
\renewcommand{\H}[2]{H_{#1}\left({#2}\right)}
\newcommand{\pn}[2]{\p\left({#1}\right)^{({#2})}}
\newcommand{\F}{{\mathcal{F}}}
\newcommand{\C}{{\mathcal{C}}}
\newcommand{\R}{{\mathbb{R}}}
\newcommand{\Z}{{\mathbb{Z}}}
\newcommand{\Q}{{\mathbb{Q}}}
\newcommand{\G}{{\mathcal{G}}}
\newcommand{\dg}{{\mathrm{dg}}}
\newcommand{\rank}{{\mathrm{rank\ }}}
\newcommand{\T}{{\mathcal{T}}}
\newcommand{\im}{{\mathrm{im\ }}}
\newcommand{\spin}{{\mathrm{spin}}}
\newcommand{\spinc}{{\mathrm{spin^c}}}
\newcommand{\N}{{\mathcal{N}}}
\newcommand{\s}{{\mathfrak{s}}}
\newcommand{\comfig}{{/Users/peter/Documents/CommonFigures/pdf}}
\renewcommand{\S}{{\mathcal{S}}}
\newcommand{\cfkinf}{CFK^\infty}


\renewcommand{\P}{{\mathcal{P}}}

% Beamer options
% ====================================================================
% ====================================================================
\usetheme{Dresden}
%\usecolortheme{wolverine}
\setbeamertemplate{theorems}[numbered]
\usecolortheme{orchid}
\usefonttheme{serif}
\begin{document}

\title[Linearly Independent TOP Slice knots]{A class of linearly independent topologically slice knots}
\author{team}
\date{July 29}
\institute{Columbia}

\frame{\titlepage}

\frame{
  \frametitle{The Concordance Group}
  \\
  Recall that two knots $K$ and $J$ are {\bf isotopic} if they can be smoothly deformed into each other. Or formally, $\exists$ a homotopy $\kappa:S^{1}\times[0,1]\longrightarrow \mathbb{R}^3$ such that $\kappa \times{0}=K$, and $\kappa \times{1}=J$, and $\kappa \times {t}$ is a smooth embedding, $\forall t\in [0,1]$. Now, we generalise this idea by considering the deformation occured in an extra dimension: we say knots $K$ and $J$ are {\bf topologically concordant/(smoothly) concordant} if they can be topologically/smoothly (resp.) deformed into each other in four dimensions. The rigorous definition is: 
}

\frame{
  \frametitle{The Concordance Group}
  \begin{def concordance}
    Oriented knots K and J are {\bf topologically concordant/(smoothly) concordant} if $\exists$ annulus $A = S^{1}\times[0,1]$ topologically/smoothly (resp.) embedded into $S^{3}\times [0,1]$, such that $A\cap (S^{3}\times {0})= K\subseteq (S^{3}\times {0})$, and  $A\cap (S^{3}\times {1})= -J\subseteq (S^{3}\times {1})$.
    \\
    Denote $K \sim_{top}J$ for topological concordance, and $K \sim_{c}J$ for (smooth) concordance).
  \end{def concordance}
}
\frame{
  \frametitle{The Concordance Group}
  \hspace{6mm}Now comes to our central (algebraic/topological) structure of interest: we note that $({Knots},\sharp) $ forms a monoid, on which $\sim_{c}$ is an equivalence relation. Hence we may define:
  \begin{def concordance}
    The {\bf Concordance Group} is $\mathcal{C}:=({Knots},\sharp)/\sim_{c}$ 
  \end{def concordance}
Note that $\forall$ knots K, -K is the mirror image of K with opposite orientation

  Also, we have the following intuitive definition:
  \begin{def concordance}
    A smooth knot concordant to the unknot is called a {\bf slice knot}. A smooth knoth topologically concordant to the unknot is called a {\bf topologically slice knot}.
  \end{def concordance}
}
\frame{
  \frametitle{Final Notes}
  $\exists$ an ordering on knots modulo concordance based on a relation $\epsilon$, to be discussed later
  
  \begin{def concordance}
    In a totally ordered group, $a$ and $b$ are said to be {\bf archimedian equivalent} if $a<n*b$ and $b<m*a$ for some $n,m \in \N$ 
  \end{def concordance}
  
  \begin{defn}
    If $a$ and $b$ are not Archimedian equivalent, and $a>b$, we say $a>>b$ (i.e. $a$ is ``much greater'' than $b$)
  \end{defn}

  \begin{prop}
    If $a_1<<a_2<<... $, then the $a_i$ are linearly independent
  \end{prop}
}

\frame{
  \frametitle{Result}
  Our main result was as follows:
  \\
  \begin{itemize}
    \item A family of Top Slice knots, all linearly independent $\tau_{(p,1)}$, where $\tau_{(p,1)}<<\tau_{(p+1,1)}$ 
  \end{itemize}
$\tau_{(p,1)} := D(T_{2,3})_{p,p+1} \sharp - T_{p,p+1}$
  \begin{itemize}
    \item A family of Top Slice knots lying between $T_{2n,2n+1}$ and $T_{2n+1,2n+2)}$ 
  \end{itemize}

}

\frame{
  \frametitle{Technique: Morphism between $\mathcal{C}$ and $\mathcal{F}$}
  How shall we study the structure of $\mathcal{C}$? We do so by forming morphisms from $\mathcal{C}$ to other groups of richer algebraic structures, and analyzing those target groups. In our research, we take the group of $\mathbb{Z}\oplus\mathbb{Z}-$filtered chain complexes, $\mathcal{F}$ (which will be discussed in detail later), and we indeed have:


}
\frame{
\begin{thm}
	There exists group homomorphism $\phi\in{Hom(\mathcal{C},\mathcal{F})}$, such that:

$(1)$ It assigns each (equivalence class of) knot a(n) (equavalence class of) chain complex; in particular, for a special class of knot, {\bf L-space knots}, such assignment is determined by the knot invariant {\bf Alexander Polynomial};

$(2)$ The group operation for $\mathcal{C}$ $(${\bf connected sum}, $\sharp)$ corresponds to the group operation for $\mathcal{F}$ $(${\bf tensor product}, $\otimes)$.

$(3)$ The (equivalence class of) {\bf reversed mirror image} of $K\in\mathcal{C}$, $[-K]$, corresponds to the {\bf dual} of $[\phi(K)]$, denoted as $[{\phi(K)}^{*}]$.
\end{thm}

}
\frame{
  \frametitle{The group of Knot Floer Complexes, $\mathcal{F}$}

  $\mathcal{F}$ is a group of rich algebraic structures: its elements are chain complexes with a natural filtration, and it is a {\bf totally ordered group}. 

  \subsection{$\mathbb{Z}\oplus\mathbb{Z}-$filtered Chain Complexes}
  \hspace{5mm} A {\bf chain complex} $(C,\partial)$ is a $\mathbb{Z}-$graded vector space with a (vs) homomophism $\partial$ between each two adjacent gradings, such that $\partial\circ\partial=0$. Hence, we have $im(\partial)\subseteq ker(\partial)$. We define the quotient group $H_{*}:=ker(\partial)/im(\partial)$ to be the {\bf homology group} of $(C,\partial)$.

  With the $\mathbb{Z}\oplus\mathbb{Z}-$filtration, we can represent the chain complex $(C,\partial)$ diagramatically on a $\mathbb{Z}\oplus\mathbb{Z}-$ lattice, using arrows to denote boudary map $\partial$.


}
\frame{
  	\frametitle{Elements of group $\mathcal{F}$}
\hspace{5mm} We can always perform the {\bf change of basis} process, which preserves the homology, to simplify the $\mathbb{Z}\oplus\mathbb{Z}-$filtered chain complex. A class of "good" complexes obtainable by change of basis, characterised by {\bf Property CFK}, consists of the elements of $\mathcal{F}$.


}

\frame{
  \frametitle{Elements of group $\mathcal{F}$}
  \begin{defn}
	A $\mathbb{Z}\oplus\mathbb{Z}-$filtered chain complex is {\bf of Property CFK} if its simplified diagram satisfies the following properties:

   $(1)$ It is symmetric;

   $(2)$ $rank_{\mathbb{F}} \Big(\bigoplus_{n} H_{*}(n^{th} column)\Big)=1$; and

   $(3)$ $rank_{\mathbb{F}} \Big(\bigoplus_{n} H_{*}(n^{th} row)\Big)=1$.
\end{defn}
	In fact, $\mathbb{Z}\oplus\mathbb{Z}-$filtered chain complexes of property CFK form a {\bf monoid} under tensor product $\otimes$.
}

\frame{
	\frametitle{$\epsilon$-equivalence}
 Now comes the main thrust: we can define an equivalence relation on the monoid (\{CFK-Chain Complexes\}, $\otimes$), which endows it with the group structure, and further induces a total order.
}

\frame{
	\frametitle{$\epsilon$-equivalence}
\begin{defn}
	Given a CFK-chain complex $(C,\partial)$, we look for the subspace $x_{0}$ supporting the nontrivial homology (whose existence and uniqueness are guaranteed by (2)(3) in the previous definition), and define:

\[
  \epsilon(C) = \left\{ 
  \begin{array}{l l}
    -1 & \quad \text{if $x_{0}$ has an outgoing horizontal arrow}\\
    1 & \quad \text{if $x_{0}$ has an incoming horizontal arrow}\\
    0 & \quad \text {if  $x_{0}$ has no horizontal arrow}\\
  \end{array} \right.
\]

	Moreover, we say two CFK-chain complexes $C$ and $C'$ are {\bf $\epsilon$-equivalent}, denoted as $C\sim_{\epsilon}C'$, if $\epsilon \big(C\otimes(C')^{*}\big)=0$.
\end{defn}


}

\frame{
In fact, $\sim_{\epsilon}$ is an equivalence relation on the CFK-chain complexes. Therefore, we obtain the group $\mathcal{F}$:


\begin{defn}
	The group of {\bf Knot Floer Homology}, $\mathcal{F}$, is defined by:
\\

\hspace{30mm}$\mathcal{F}$:=\big(\{CFK-chain complexes\}, $\otimes$\big)/$\sim_{\epsilon}$
\end{defn}


	It's easy to see that $\mathcal{F}$ is indeed a group, with trivial chain complex as the identity, and the dual chain complex as the inverse. 

}





\frame{
	\frametitle{Ordering on $\mathcal{F}$}

\hspace{5mm} We have remarked that the $\epsilon$-equivalence induces a total order on $\mathcal{F}$. Indeed, 

\begin{defn}
	$\big(\mathcal{F},\leq\big)$ is a totally ordered group, with the order $\leq$ given by:

	\[
   \left\{ 
  \begin{array}{l l}
    [C] < [C'] & \quad \text{if  $\epsilon \big( C\otimes (C')^{*}\big)=-1$}\\
    $[C] = [C']$ & \quad \text {if  $C\sim_{\epsilon}C'$}\\
  \end{array} \right.
\]
\end{defn}


}
\end{document}
	
